\documentclass[a4paper]{article}
\usepackage[letterpaper, margin=1in]{geometry} % page format
\usepackage{listings} % this package is for including code
\usepackage{graphicx} % this package is for including figures
\usepackage{amsmath}  % this package is for math and matrices
\usepackage{amsfonts} % this package is for math fonts
\usepackage{tikz} % for drawings
\usepackage{hyperref} % for urls
\usepackage{pdfpages}

\title{Project Propsal}
\author{Max Schemitsch}
\date{2/13/2019}

\begin{document}
\lstset{language=Python}

\maketitle

\section{Project Description}

For my project, I want to focus on the impact of machine learning on social media platforms. Companies are keen on recognizing patterns across different types of media such as text, images, audio, video, etc; such types of data are called "unstructured" as opposed to structured data seen in databases with rows and columns. Structured data sets are easier to categorize into different groups (e.g. something like the Perceptron Learning Algorithm). While it is apparent that companies like Facebook identify and utilize user interests to select what media is displayed, I would like to use this project to go further in-depth with how this works. Additionally, I would like to cover the future of the cross-section between machine learning and social media, and where it is headed.
\\

There are a multitude of topics that I would like to look at for this project, but there are a few that are more important than the rest. Sentiment analysis relates to what people are saying about a certain topic. This primarily comes from text based posts similar to those seen on Twitter, Facebook, Instagram, etc. Audience analysis is about identifying group characteristics related to certain topics. An example would be what age group is most likely to be buying a certain pharmaceutical or medical product. Image analysis takes apart images, a format without any text, and can identify company logos, human faces, scenery, and objects. Google and Apple both utilize this type of recognition in their applications.
\\

On top of these general topics, I hopefully will be able to find information about the actual algorithms used at these companies. Ideally my project will be a mix of both machine learning theory and practicality. As I work through the project I expect that I will uncover more topics to work on that I haven't thought of yet.
\end{document}